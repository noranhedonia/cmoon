\subsection{Wayland}
The Wayland book is a great resource to figure out details of how to implement Wayland clients.
[wayland-book]\url{https://wayland-book.com/introduction.html}

The libwayland-client and libwayland-server libraries implement the wire protocol for each end of the connection.
These libraries also include some utilities for working with Wayland data structures, a simple event loop, and a pre-compiled copy of the core Wayland protocol.
But we will not be using libwayland directly, implementing a connection with the Wayland compositor ourselves via the wire protocol.

There's also a great presentation in Zig made by Jonathan Marler for the X11 server, touching basically the same principles we'll be exploring via our Wayland backend (and X11 later).
[How to Use Abstraction to Kill Your API - Jonathan Marler - Software You Can Love Vancouver 2023]\url{https://www.youtube.com/watch?v=aPWFLkHRIAQ}

\subsubsection{The wire protocol}
[The Wayland Protocol, by Kristian Høgsberg]\url{https://wayland.freedesktop.org/docs/html/ch04.html#sect-Protocol-Wire-Format} \\
[The Wayland Protocol, by Drew DeVault]\url{https://wayland-book.com/protocol-design/wire-protocol.html}

The protocol is a stream of 32-bit values, encoded with a native endianess.
Using Wire we represent values as the following primitive types:
\begin{itemize}
    \item \textbf{int, uint}: 32-bit signed or unsigned integer.
    \item \textbf{fixed}: 24.8-bit signed fixed-point numbers.
    \item \textbf{object}: 32-bit object ID.
    \item \textbf{new\_id}: 32-bit object ID which allocates that object when received.
    \item \textbf{string}: Prefixed with a 32-bit integer specifying its length (in bytes),
        followed by the string contents and a NUL terminator, padded to 32 bits with undefined data.
        We'll be using the UTF-8 encoding.
    \item \textbf{array}: Prefixed with a 32-bit integer specifying its length (in bytes),
        then the verbatim contents of the array, padded to 32 bits with undefined data.
    \item \textbf{fd}: transfers a file descriptor to the other end using the ancillary data 
        in the UNIX domain socket message (msg\_control).
    \item \textbf{enum}: A single value (or bitmap) from an enumeration of known constants, encoded into a 32-bit integer.
\end{itemize}

