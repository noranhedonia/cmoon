\section{Ver. 0.0, revisions 1-X}

\subsection{Sorceress}
TODO framework

\subsubsection{Fibers and the job system}
This fiber-based job system is implemented based on ideas presented by Naughty Dog
in the GDC2025 talk ``Parallelizing the Naughty Dog Engine using Fibers''.

[Naughty Dog Video] 
\url{http://gdcvault.com/play/1022186/Parallelizing-the-Naughty-Dog-Engine}

Running this multithreaded, the work is not guaranteed to be executed in the order it was submitted.
That's the deal, for work to run in parallel. 
Yielding jobs doesn't ``block'' in the traditional sense, instead the current context of execution
is switched with a fiber that contains details of a new job to run.
These details include a function pointer, data as an argument to the function, 
a hint for how large the fiber stack should be and a name of the job for the fiber to inherit.
To yield one must acquire a chain. 
This chain can be bound to a submission, then the yielding fiber may not resume until all work bound to the chain is done.

Context switching is very CPU specific, so this functionality is written in assembly.
An implementation must be provided for every pair or CPU architecture, platform ABI and assembler to be supported.
A great deal of help in implementing this was the source code to Boost C++ fiber context library.
It has most of this stuff figured out for a lot of different targets.

[Boost::Context]
\url{https://github.com/boostorg/context}

[Fibers, Oh My!]
\url{https://graphitemaster.github.io/fibers/}

In my last implementation of this job system in C, fiber stacks have been preallocated as a large flat array.
There was a lot of problems with this design. 
The memory usage was a large one, fiber stacks had to be sized for worst case.
For example a stack size of 100 KiB was not enough to call into vkCreateInstance().
It caused a segfault by corrupting the stacks of neighbouring fibers.
So for a system with 160 fibers running, every fiber would need atleast 128 KiB for the needs of the application,
but only really few jobs needed this much stack space to work -- mostly calls to heavy libraries or drivers.

Here I want to improve on that by making the fiber stacks dynamically allocated.
With that we'll be able to size every job for just as much stack space it needs to run.
I'll achieve this by implementing an allocator, here called \textit{drifter}, for transient resources bound to a fiber context.
Generally we make a double-ended stack where one end is for user-requested heap allocations (works like an arena allocator),
and the second end would be responsible for slicing stack spaces for fibers.
If a fiber yields, there are cases where the next fiber may inherit the drifter and make use of some unused stack space.
A drifter imposes their own challenges but we'll tackle them later.

Fundamental to the fiber code is thread local storage (TLS). 
We are rolling out an array of such structures and assign them to threads by an \texttt{0..thread\_count} index.
The thread index runs on Zig's \texttt{threadlocal}. 

A deadlock happens if the MPMC ring buffer is full and running from only one thread.
This will also happen when dequeuing from an empty buffer on singlethreaded applications.
The waiting for enqueue/dequeue thread may never leave the loop 
when there is no other thread to try to empty/fill the buffer. 
This should not really be an issue unless stress testing the job system on singlethreaded environments.

May get into improving context switching prediction, if this can improve speed:
\url{http://www.crystalclearsoftware.com/soc/coroutine/coroutine/linuxasm.html}

\subsubsection{Drifter allocator}
A large problem of the drifter comes from a nature of fiber-based systems. The fiber migrates.
There is no guarantee, that the thread that yields a fiber will be the same thread to resume it.
We can't store this allocator in TLS, as it must migrate together with fibers, in deterministic ways.
We'll we can implement such safety measure, but there wouldn't be a point in running fibers then, right?

Drifter transitions can be identified by taking the state of both FROM and TO fibers:
NO drifter, we are the OWNER of the drifter, we INHERITED the drifter, and the yield relationship between them (WAIT or FREE). 
These terms I guess mean:
\begin{itemize}
    \item NO: The FROM fiber can be in this state at initialization or in the race condition fix (check \texttt{Sorceress.Internal.yieldFiber()}). 
        The TO fiber is always in this state when we grab a free fiber and a new job.
    \item OWNER: The fiber has direct ownership over the drifter. At FREE it can pass it down to another fiber without conflicts. Yield to WAIT can be tricky.
    \item INHERITED: The fiber has no direct ownership, as it has inherited a drifter previously used by one or more waiting fibers. 
        Passing the drifter around can generate conflicts that we must resolve first.
    \item WAIT: The FROM fiber is added to the waitlist, the TO fiber will have to either inherit or in another way transfer the drifter without conflicts with the waiting context.
    \item FREE: The FROM fiber is done with it's work and will be released, the TO fiber should inherit or orphan the drifter.
\end{itemize}

\subsection{Wayland}
The Wayland book is a great resource to figure out details of how to implement Wayland clients.
[wayland-book]\url{https://wayland-book.com/introduction.html}

The libwayland-client and libwayland-server libraries implement the wire protocol for each end of the connection.
These libraries also include some utilities for working with Wayland data structures, a simple event loop, and a pre-compiled copy of the core Wayland protocol.
Even though we have a low-level wayland protocol implementation at our hands, we still need to make things compatible with libwayland-client,
because Vulkan drivers still expect us to use it -- if we wan't to use the swapchain, that is.

There's also a great presentation in Zig made by Jonathan Marler for the X11 server, touching basically the same principles we'll be exploring via our Wayland backend (and X11 later).
[How to Use Abstraction to Kill Your API - Jonathan Marler - Software You Can Love Vancouver 2023]\url{https://www.youtube.com/watch?v=aPWFLkHRIAQ}

\subsubsection{The wire protocol}
[The Wayland Protocol, by Kristian Høgsberg]\url{https://wayland.freedesktop.org/docs/html/ch04.html#sect-Protocol-Wire-Format} \\
[The Wayland Protocol, by Drew DeVault]\url{https://wayland-book.com/protocol-design/wire-protocol.html}

The protocol is a stream of 32-bit values, encoded with a native endianess.
Using Wire we represent values as the following primitive types:
\begin{itemize}
    \item \textbf{int, uint}: 32-bit signed or unsigned integer.
    \item \textbf{fixed}: 24.8-bit signed fixed-point numbers.
    \item \textbf{object}: 32-bit object ID.
    \item \textbf{new\_id}: 32-bit object ID which allocates that object when received.
    \item \textbf{string}: Prefixed with a 32-bit integer specifying its length (in bytes),
        followed by the string contents and a NUL terminator, padded to 32 bits with undefined data.
        We'll be using the UTF-8 encoding.
    \item \textbf{array}: Prefixed with a 32-bit integer specifying its length (in bytes),
        then the verbatim contents of the array, padded to 32 bits with undefined data.
    \item \textbf{fd}: transfers a file descriptor to the other end using the ancillary data 
        in the UNIX domain socket message (msg\_control).
    \item \textbf{enum}: A single value (or bitmap) from an enumeration of known constants, encoded into a 32-bit integer.
\end{itemize}

\subsection
